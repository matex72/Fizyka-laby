\documentclass[12pt]{article}
\usepackage{polski}
\usepackage[utf8]{inputenc}
\usepackage{graphicx}
\usepackage{amsmath}
\usepackage{graphicx}
\usepackage{setspace}
\usepackage{pdfpages}


\title{ Badanie zjawiska Peltiera \\
    \large Informatyka – profil praktyczny, semestr II \\
    Wydział Matematyki Stosowanej \\
    Politechnika Śląska \\}

\author{ Sekcja 5 \\
    Piotr Skowroński, Bartłomiej Pacia}
\date{Kwiecień 2022}

\begin{document}

\maketitle

\section{Wstęp teoretyczny}

Efekt Peltiera to zjawisko termoelektryczne występujące ciałach stałych. Polega
na wydzielaniu lub pochłanianiu przez złącze energii pod wpływem przepływu prądu
elektrycznego.

Zostało po raz pierwszy zaobserwowane i opisane w 1834 roku przez francuskiego
fizyka Jeana Peltiera. Jest zjawiskiem odwrotnym do efektu Seebecka, który
został odkryty w 1821 roku.

Aby zaszedł efekt Peltiera, potrzeba złączyć dwa różne przewodniki lub
półprzewodniki dwoma złączami, tzw. \textit{złączami Peltiera}. Podczas
przepływu prądu złącze, w którym elektrony przechodzą z przewodnika o niższym
poziomie Fermiego do przewodnika o wyższym, ulega ochłodzeniu. Drugie złącze
ulega w tym czasie ogrzewaniu. Po zmianie kierunku na przeciwny efekt ulega
odwróceniu.

\section{Pomiary}

Podczas wykonywania doświadczenia w pracowni pomiary zapisywaliśmy ręcznie na
kartce. Następnie przepisaliśmy wyniki naszych pomiarów do pliku CSV, by
umożliwić ich wykorzystanie w programie.

Użyliśmy języka Python w środowisku Jupyter Notebook. Wykorzystaliśmy biblioteki
\textit{numpy}, \textit{pandas} i \textit{matplotlib}.

\section{Obliczenia i wykresy}

\subsection*{Zadanie 1. Obliczenie różnicy temperatur $\Delta T$ wygenerowaną
    przez element Peltiera dla każdego prądu}

todo

\section{Wnioski}

tbh to nie wiem. xD

\end{document}
