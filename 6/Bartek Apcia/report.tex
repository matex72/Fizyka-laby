\documentclass[12pt]{article}
\usepackage{polski}
\usepackage[utf8]{inputenc}
\usepackage{graphicx}
\usepackage{amsmath}
\usepackage{graphicx}
\usepackage{float}
\graphicspath{ {./images/} }
\usepackage{setspace}
\onehalfspacing

\title{ Wyznaczenie długości fali światła laserowego przy pomocy siatki
dyfrakcyjnej \\
    \large Informatyka – profil praktyczny, semestr II \\
    Wydział Matematyki Stosowanej \\
    Politechnika Śląska \\}

\author{ Sekcja 5 \\
    Piotr Skowroński, Bartłomiej Pacia}
\date{Marzec 2022}

\begin{document}

\maketitle

\section{Wstęp teoretyczny}

Od początku XX wieku wiemy, że światło ma naturę dualną. Oznacza to, że zjawiska
charakterystyczne dla fal, jak np. interferencja i dyfrakcja, wyjaśnia się za
pomocą teorii falowej. Do zjawisk, których teoria falowa nie jest w stanie
wyjaśnić, używa się teorii korpuskularnej.

Jednym z parametrów charakteryzujących falę jest jej długość. Jest to
najmniejsza odległość między dwoma punktami o tej samej fazie drgań.

\begin{figure}[H]
    \centering
    \includegraphics[width=0.8\textwidth]{waves.png}
    \caption{Wizualizacja długości fali. Oznacza się ją grecką literą $\lambda$ (lambda). }
\end{figure}
Długości fal świetlnych, które widzimy, mają długość od 380 do 780 nm.
Siatka dyfrakcyjna jest to przyrząd, służący do przeprowadzenia analizy widmowej światła.

Zbudowana jest z układu równych, równoległych i jednakowo rozmieszczonych szczelin.
Działanie siatki polega na wykorzystaniu zjawiska dyfrakcji i interferencji światła do uzyskania jego widma.
W tym celu pomiędzy źródłem światła a ekranem umieszcza się siatkę dyfrakcyjną. Na ekranie uzyskuje się w ten sposób widmo światła.

\section{Pomiary}

Podczas wykonywania doświadczenia w pracowni pomiary zapisywaliśmy ręcznie na
kartce. Następnie przepisaliśmy wyniki naszych pomiarów do pliku JSON, by
umożliwić ich wykorzystanie w programie.

Do obliczeń wykorzystaliśmy język Python w środowisku Jupyter Notebook.

\section{Obliczenia}

\subsection*{Zadanie 1. Obliczenie średniej wartości $x_N$ dla każdej pary $x_L$
    i $x_P$.} Przyjmiemy niepewności $u(x_L)=0.003m$, $u(x_P)=0.003m$ i $u(L)=0.003m$, związane
z niepewnością linijki. Wartości średnich liczymy ze wzoru $x_N = \frac{x_L + x_P}{2}$ dla każdej pary $x_L$, $x_P$.

Niepewność $u(x_N)$ wyznaczamy z prawa przenoszenia niepewności:
\begin{center}
    $u(x_N) = \sqrt{[\frac{\partial x_N}{\partial x_L}u(x_L)]^2 + [\frac{\partial x_N}{\partial x_P}u(x_P)]^2}$ \\
\end{center}

Po obliczeniach:
\begin{center}
    $u(x_N) = \frac{1}{2}\sqrt{u^2(x_L) + u^2(x_P)} = 0.0021$ m
\end{center}

\begin{center}
    \begin{tabular} { | c | c | c | c | c | c | }
        \hline
        Lp. & $N$ & $L$, m   & $x_L$, m & $x_P$, m & $x_N$, m   \\
        \hline
        1.  & 1   & 0.100(3) & 0.018(3) & 0.018(3) & 0.0180(21) \\ \hline
        2.  & 2   & 0.100(3) & 0.040(3) & 0.039(3) & 0.0395(21) \\ \hline
        3.  & 3   & 0.100(3) & 0.066(3) & 0.065(3) & 0.0655(21) \\ \hline
        4.  & 4   & 0.100(3) & 0.106(3) & 0.110(3) & 0.1080(21) \\ \hline
        5.  & 1   & 0.200(3) & 0.039(3) & 0.038(3) & 0.0385(21) \\ \hline
        6.  & 2   & 0.200(3) & 0.081(3) & 0.080(3) & 0.0805(21) \\ \hline
        7.  & 3   & 0.200(3) & 0.136(3) & 0.136(3) & 0.1360(21) \\ \hline
        8.  & 1   & 0.300(3) & 0.058(3) & 0.058(3) & 0.0580(21) \\ \hline
        9.  & 2   & 0.300(3) & 0.122(3) & 0.123(3) & 0.1230(21) \\ \hline
        10. & 3   & 0.300(3) & 0.205(3) & 0.215(3) & 0.2100(21) \\ \hline
        11. & 1   & 0.400(3) & 0.078(3) & 0.077(3) & 0.0755(21) \\ \hline
        12. & 2   & 0.400(3) & 0.166(3) & 0.164(3) & 0.1650(21) \\ \hline
        13. & 1   & 0.500(3) & 0.097(3) & 0.097(3) & 0.0970(21) \\ \hline
        14. & 2   & 0.500(3) & 0.205(3) & 0.216(3) & 0.2110(21) \\
        \hline
    \end{tabular}
\end{center}

\subsection*{Zadanie 2. Obliczenie długości fali światła laserowego dla każdego $x_N$}

Aby obliczyć długość fali światła laserowego skorzystamy ze wzoru:
\begin{center}
    $\lambda = \frac{d}{N} \cdot \frac{x_N}{\sqrt{x_N^2+L^2}}$
\end{center}
Gdzie:

$d$ - stała siatki dyfrakcyjnej

$L$ - odległość siatki od ekranu

$N$ - rząd prążka dyfrakcyjnego

$d = \frac{1mm}{\text{liczba nacięć na siatce na 1mm}} = \frac{1mm}{300} =
    \frac{1}{300000}$ m


\begin{center}
    \begin{tabular} { | c | c | c | c | c | }
        \hline
        Lp. & N  & $L$, m   & $x_N$, m   & $\lambda$, nm \\
        \hline
        1.  & 1  & 0.100(3) & 0.0180(21) & 591(73)       \\ \hline
        2.  & 2  & 0.100(3) & 0.0395(21) & 612(39)       \\ \hline
        3.  & 3  & 0.100(3) & 0.0655(21) & 609(25)       \\ \hline
        4.  & 4  & 0.100(3) & 0.1080(21) & 611(15)       \\ \hline
        5.  & 1  & 0.200(3) & 0.0385(21) & 630(37)       \\ \hline
        6.  & 2  & 0.200(3) & 0.0805(21) & 622(19)       \\ \hline
        7.  & 3  & 0.200(3) & 0.1360(21) & 625(13)       \\ \hline
        8.  & 1  & 0.300(3) & 0.0580(21) & 633(25)       \\ \hline
        9.  & 2  & 0.300(3) & 0.1230(21) & 630(13)       \\ \hline
        10. & 3  & 0.300(3) & 0.2100(21) & 637(10)       \\ \hline
        11. & 1. & 0.400(3) & 0.0755(21) & 634(18)       \\ \hline
        12. & 2. & 0.400(3) & 0.1650(21) & 636(10)       \\ \hline
        13. & 1. & 0.500(3) & 0.0970(21) & 635(15)       \\ \hline
        14. & 2. & 0.500(3) & 0.2110(21) & 647(11)       \\
        \hline
    \end{tabular}
\end{center}

\subsection*{Zadanie 3. Obliczenie niepewności długości fali $u(\lambda$).}
Aby wyliczyć niepewności długości fal, skorzystamy z prawa przenoszenia
niepewności. Przyjmujemy, że niepewność L wynosi $u(L) = 0.003$m. Prawo
przenoszenia niepewności wyraża się wzorem:

\begin{center}
    $u(y)=\sqrt{\sum_{i=1}^{k}[\frac{\partial y}{\partial x_i}u(x_i)]^2}$
\end{center}
Zatem prawo przenoszenia niepewności dla $\lambda$ ma postać:
\begin{center}

    $u(\lambda) = \sqrt{[\frac{\partial \lambda}{\partial x_n}u(x_n)]^2 +
        [\frac{\partial \lambda}{\partial L}u(L)]^2}$
\end{center}
Po uproszczeniu:
\begin{center}
    $u(\lambda) = \sqrt{[\frac{dL^2u(x_N)}{N(x_N^2+L^2)^{1.5}}]^2  +
        [\frac{-dLx_Nu(L)}{N(x_N^2+L^2)^{1.5}}]^2}$
\end{center}
Niepewności pomiarowe dla $\lambda$ wprowadziliśmy od razu do tabelki powyżej.

\subsection*{Zadanie 4. Obliczenie średniej ważonej z $\lambda$.}
Wzór na średnią ważoną ma postać:
\begin{center}
    $\bar{\lambda} = \frac{\sum_{i=1}^{N}\lambda_iw_i}{\sum_{i=1}^{N}w_i}$,
    gdzie $w_i = \frac{1}{u^2(\lambda_i)}$
\end{center}
Zatem po zliczeniu wszystkich danych otrzymujemy:
\begin{center}
    $\bar{\lambda} = 634$ nm
\end{center}

\subsection*{Zadanie 5. Obliczenie niepewności średniej ważonej}
Wzór na niepewność średniej ważonej ma postać:
\begin{center}
    $u(\bar{\lambda}) = \frac{1}{\sqrt{\sum_{i=1}^{N}w_i}}$
\end{center}
Po obliczeniach:
\begin{center}
    $u(\bar{\lambda}) = 3.7$ nm
\end{center}
\subsection*{Zadanie 6. Zapisanie wyników w odpowiednim formacie}
Końcowa wartość $\lambda$ ma wartość średniej ważonej wraz z jej niepewnością:
\begin{center}
    $\lambda = 634.0(3.7) $ nm
\end{center}

\subsection*{7. Wnioski}
Zmierzona przez nas długość fali światła laserowego $\lambda$ odpowiada kolorowi czerwonemu.
Zgadza się to z faktycznym kolorem lasera w pracowni. \\
Końcowe niepewności pomiarów wynikają z niepewności linijki użytej do pomiaru
odległości siatki od ekranu, jak również z nieprawidłowego odczytu odległości prążków od prążka zerowego.
\end{document}
